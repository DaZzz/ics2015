\documentclass{lutmscthesis}[2010/09/22]

\usepackage[latin1]{inputenc}
\usepackage[T1]{fontenc}
\usepackage[english,finnish]{babel}

\usepackage{times}

\usepackage{setspace}
\usepackage{verbatim}
\usepackage[intlimits]{amsmath}

% Ensure figure captions are below and table captions are above the content.
\usepackage{float}
\floatstyle{plain}\restylefloat{figure}
\floatstyle{plaintop}\restylefloat{table}

\usepackage[pdfborder={0 0 0}]{hyperref}


\graphicspath{{resources/images/}}                % Graphics search path

\newcommand{\vect}[1]{\boldsymbol{#1}}
\newcommand{\matr}[1]{\boldsymbol{#1}}
\newcommand{\diag}[1]{\mathrm{diag}(#1)}
\newcommand{\iprod}[1]{\left\langle #1 \right\rangle}
\newcommand{\me}{\mathrm{e}}
\newcommand{\mi}{\mathrm{i}}
\newcommand{\md}{\mathrm{d}}
\newcommand{\sse}{{}} %\mathrm{SSE}}
\newcommand{\trace}{\mathrm{Tr}\:}
\newcommand{\frp}[2]{{}^\mathrm{#1}\vect{#2}}
\newcommand{\frs}[3]{{}^\mathrm{#1}#2_\mathrm{#3}}
\newcommand{\frv}[3]{{}^\mathrm{#1}\vect{#2}_\mathrm{#3}}
\newcommand{\frm}[3]{{}^\mathrm{#1}\matr{#2}_\mathrm{#3}}
\newcommand{\colvec}[2]{\genfrac{[}{]}{0pt}{1}{#1}{#2}}
\newcommand{\relphantom}[1]{\mathrel{\phantom{#1}}}

% Tables
\usepackage{array}
\usepackage{multirow}
\newcolumntype{L}[1]{>{\raggedright\let\newline\\\arraybackslash\hspace{0pt}}m{#1}}
\newcolumntype{C}[1]{>{\centering\let\newline\\\arraybackslash\hspace{0pt}}m{#1}}
\newcolumntype{R}[1]{>{\raggedleft\let\newline\\\arraybackslash\hspace{0pt}}m{#1}}

% Code
\usepackage{listings}
\usepackage{xcolor}
\renewcommand{\lstlistingname}{Listing}

\lstset{frame=single,
  language=Python,
  aboveskip=10mm,
  belowskip=0mm,
  captionpos=b,
  basicstyle={\small\ttfamily},
  numbers=none,
  breaklines=false,
  breakatwhitespace=true,
  framexleftmargin=6pt,
  framexrightmargin=6pt,
  framextopmargin=4pt,
  framexbottommargin=4pt
}


% Thesis information
\title{OPPONENT REVIEW FOR THE REPORT:
THE ROLE OF INTELLIGENT COMPUTING IN FAULT DETECTION OF PRINTED CIRCUIT BOARDS}
\author{Review author: Tikhon Belousko \\
Report author: Joni Herttuainen}
\Major{Degree Program in Intelligent Computing}
\Faculty{School of Engineering Science}
\Doctype{}
\Keywords{visual inspection, computer vision, machine vision, recycling}
\Supervisor{Leena Ikonen D.Sc. (Tech.)}
\Examiner{Leena Ikonen D.Sc. (Tech.)}
\Year{2016}


% ---
% Document
% ---
\begin{document}
\selectlanguage{english}

% ---
% Title
% ---
\maketitle
\newpage

% ---
% Review
% ---
\section{ GENERAL REVIEW }
\setlength{\parskip}{3ex}

In his work Joni Herttuainen attempted to summarize modern methods used
for detection of defects and faults in printed circuit boards production.
The reasoning why the problem is important has been presented clearly.
The report is well structured, easy to understand and read.
Four different methods were described for PCB fault detection that
were developed in recent years. Also, author covered three
main targets of inspection.

This work could be useful for students who study Machine Vision and want
to deeper understand possible applications of machine vision systems.
The report may also can be suitable for
researchers who start to work on topic of circuit board visual inspection.

Although, Joni covered several methods for PCB fault detection in concise
manner, the article can still be improved by adding more examples and details for
each of the methods. Some of the techniques are introduced without an explanation
so it would be hard to understand all presented concepts without strong background.
The article may also benefit from adding more information about the results of
each of the presented methods.

% ---
% Evaluation
% ---
\section{ EVALUATION }

The evaluation according to the Master Thesis criteria is given in the Table~\ref{tab:eval}.

\begin{table}[hpt]
\begin{center}
\caption{Evaluation and grading.\label{tab:eval}}
{\renewcommand{\arraystretch}{2}
\begin{tabular}{| l | c |}

% Title
\hline
\textbf{Criteria}
&
\textbf{Grade} \\
\hline

Definition of research problem, objectives, and delimitations. & 5 \\
\hline
Research approach, methods and materials. & 5 \\
\hline
Utilization of existing research knowledge. & 4 \\
\hline
Systematic and responsible execution of the project. & 4 \\
\hline
Logic and credibility of the interpretation of the results and the conclusions. & 4 \\
\hline
Usability of the results. & 4 \\
\hline
Readability, presentation and language of the report. & 4 \\
\hline
\textbf{Criteria} & 4 \\
\hline


\end{tabular}}
\end{center}
\end{table}





\end{document}
